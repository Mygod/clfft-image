\documentclass{article}
\usepackage{amsmath, color, listings, mathrsfs}

\definecolor{dkgreen}{rgb}{0,0.6,0}
\definecolor{gray}{rgb}{0.5,0.5,0.5}
\definecolor{mauve}{rgb}{0.58,0,0.82}

\lstset{
frame=tb,
  language=C,
  showstringspaces=false,
  columns=flexible,
  basicstyle={\small\ttfamily},
  numbers=none,
  numberstyle=\tiny\color{gray},
  keywordstyle=\color{blue},
  commentstyle=\color{dkgreen},
  stringstyle=\color{mauve},
  breaklines=true,
  breakatwhitespace=true
}

\begin{document}

  \title{Parallel FFT with OpenCL}

  \maketitle

  This program does the following:

  \begin{itemize}
  \item Take in a 2-dimensional PNG image, convert its colorspace to 24-bit RGB if it's not;
  \item Multiply every pixel with $\left(-1\right)^{\left(x + y\right)}$ to move the low frequency part to the center
        of the image, since $ \mathscr{F}\left(f\left(x, y\right) \left(-1\right)^{\left(x + y\right)}\right)
         = \mathscr{F}\left(u - \frac{M}{2}, v - \frac{N}{2}\right) $.

        Proof:
        \begin{equation*}
        \begin{aligned}
        \mathscr{F}\left(f\left(x, y\right) \left(-1\right)^{x + y}\right)\left(u, v\right)
          &= \sum\limits_{x = 1}^N \sum\limits_{y = 1}^M f\left(x, y\right) \left(-1\right)^{x + y}
             e^{-2 \pi i \frac{ux + vy}{MN}} \\
          &= \sum\limits_{x = 1}^N \sum\limits_{y = 1}^M f\left(x, y\right) e^{\pi i \left(x + y\right)}
             e^{-2 \pi i \frac{ux + vy}{MN}} \\
          &= \sum\limits_{x = 1}^N \sum\limits_{y = 1}^M f\left(x, y\right)
             e^{-2 \pi i \frac{\left(u - \frac{M}{2}\right)x + \left(v - \frac{N}{2}\right)y}{MN}} \\
          &= \mathscr{F}\left(u - \frac{M}{2}, v - \frac{N}{2}\right)\left(u, v\right).
        \end{aligned}
        \end{equation*}
  \item Do a fast fourier transform on each channel (R, G and B) in the transformed image after zero-padding the image
        so that width and length of the image are both powers of 2, 3, 5 and 7 (limited by clFFT library). For example,
        an image of size 1300x1300 will be zero-padded to 1323x1323, since $ 1323 = 3^3 \times 7^2 $.
  \item The result is visualized using the following equation before saving the new PNG to disk:
        $$ f'\left(x, y\right) =
            255 \log_{256} \left(1 + \frac{\left|\mathscr{F}\left(x, y\right)\right|}{MN}\right). $$
  \end{itemize}


  \section{Advantages of using OpenCL}

  OpenCL is a framework for writing programs that execute across heterogeneous platforms consisting of central
  processing units (CPUs), graphics processing units (GPUs), digital signal processors (DSPs), field-programmable gate
  arrays (FPGAs) and other processors or hardware accelerators. And thus one can take advantage of all the available
  computing resources on a computer without the hassle of writing code individually for each device. OpenCL uses a
  programming language that is based on C99 so it's also faster to get started.

  Here is the kernel function for preparing 2D FFT data from one of the channels in the bitmap and visualizing the
  output FFT gives.

  \begin{lstlisting}
__kernel void inputX(__global char *bitmap, __global float2 *in,
                       int width) {
  int x = get_global_id(0), y = get_global_id(1), i = y * width + x;
  in[i] = (float2) ((x + y) & 1 ? -(float) bitmap[i * 3] : bitmap[i * 3],
                      0);
}

__kernel void outputX(__global char *bitmap, __global float2 *out,
                        int width) {
  int x = get_global_id(0), y = get_global_id(1), i = y * width + x;
  bitmap[i * 3] = (char) 105.886458025 *
    log(1 + sqrt(out[i].x * out[i].x + out[i].y * out[i].y));
}
  \end{lstlisting}

  OpenCL compiles these kernel functions and run-time but it's only needed once for every run and takes less than 1
  second so its impact can be ignored.


  \section{Performance comparison between CPU and GPU}

  One can query all the available OpenCL devices with a utility called clinfo.

  \begin{lstlisting}[language={}]
$ clinfo
Number of platforms                               2
  Platform Name                                   Intel(R) OpenCL
  Platform Vendor                                 Intel(R) Corporation
  Platform Version                                OpenCL 1.2 LINUX
  Platform Profile                                FULL_PROFILE
  Platform Extensions                             cl_khr_icd cl_khr_global_int32_base_atomics cl_khr_global_int32_extended_atomics cl_khr_local_int32_base_atomics cl_khr_local_int32_extended_atomics cl_khr_byte_addressable_store cl_khr_depth_images cl_khr_3d_image_writes cl_intel_exec_by_local_thread cl_khr_spir cl_khr_fp64 
  Platform Extensions function suffix             INTEL

  Platform Name                                   NVIDIA CUDA
  Platform Vendor                                 NVIDIA Corporation
  Platform Version                                OpenCL 1.1 CUDA 6.5.51
  Platform Profile                                FULL_PROFILE
  Platform Extensions                             cl_khr_byte_addressable_store cl_khr_icd cl_khr_gl_sharing cl_nv_compiler_options cl_nv_device_attribute_query cl_nv_pragma_unroll cl_nv_copy_opts 
  Platform Extensions function suffix             NV

  Platform Name                                   Intel(R) OpenCL
Number of devices                                 1
  Device Name                                     Intel(R) Core(TM) i5-3230M CPU @ 2.60GHz
  Device Vendor                                   Intel(R) Corporation
  Device Vendor ID                                0x8086
  Device Version                                  OpenCL 1.2 (Build 10002)
  Driver Version                                  1.2.0.10002
  Device OpenCL C Version                         OpenCL C 1.2 
  Device Type                                     CPU
  Device Profile                                  FULL_PROFILE
  Max compute units                               4
  Max clock frequency                             2600MHz
  Device Partition                                (core)
    Max number of sub-devices                     4
    Supported partition types                     by counts, equally, by names (Intel)
  Max work item dimensions                        3
  Max work item sizes                             8192x8192x8192
  Max work group size                             8192
  Preferred work group size multiple              128
  Preferred / native vector sizes                 
    char                                                 1 / 16      
    short                                                1 / 8       
    int                                                  1 / 4       
    long                                                 1 / 2       
    half                                                 0 / 0        (n/a)
    float                                                1 / 8       
    double                                               1 / 4        (cl_khr_fp64)
  Half-precision Floating-point support           (n/a)
  Single-precision Floating-point support         (core)
    Denormals                                     Yes
    Infinity and NANs                             Yes
    Round to nearest                              Yes
    Round to zero                                 No
    Round to infinity                             No
    IEEE754-2008 fused multiply-add               No
    Support is emulated in software               No
    Correctly-rounded divide and sqrt operations  No
  Double-precision Floating-point support         (cl_khr_fp64)
    Denormals                                     Yes
    Infinity and NANs                             Yes
    Round to nearest                              Yes
    Round to zero                                 Yes
    Round to infinity                             Yes
    IEEE754-2008 fused multiply-add               Yes
    Support is emulated in software               No
    Correctly-rounded divide and sqrt operations  No
  Address bits                                    64, Little-Endian
  Global memory size                              8252817408 (7.686GiB)
  Error Correction support                        No
  Max memory allocation                           2063204352 (1.922GiB)
  Unified memory for Host and Device              Yes
  Minimum alignment for any data type             128 bytes
  Alignment of base address                       1024 bits (128 bytes)
  Global Memory cache type                        Read/Write
  Global Memory cache size                        262144
  Global Memory cache line                        64 bytes
  Image support                                   Yes
    Max number of samplers per kernel             480
    Max size for 1D images from buffer            128950272 pixels
    Max 1D or 2D image array size                 2048 images
    Max 2D image size                             16384x16384 pixels
    Max 3D image size                             2048x2048x2048 pixels
    Max number of read image args                 480
    Max number of write image args                480
  Local memory type                               Global
  Local memory size                               32768 (32KiB)
  Max constant buffer size                        131072 (128KiB)
  Max number of constant args                     480
  Max size of kernel argument                     3840 (3.75KiB)
  Queue properties                                
    Out-of-order execution                        Yes
    Profiling                                     Yes
    Local thread execution (Intel)                Yes
  Prefer user sync for interop                    No
  Profiling timer resolution                      1ns
  Execution capabilities                          
    Run OpenCL kernels                            Yes
    Run native kernels                            Yes
    SPIR versions                                 1.2
  printf() buffer size                            1048576 (1024KiB)
  Built-in kernels                                
  Device Available                                Yes
  Compiler Available                              Yes
  Linker Available                                Yes
  Device Extensions                               cl_khr_icd cl_khr_global_int32_base_atomics cl_khr_global_int32_extended_atomics cl_khr_local_int32_base_atomics cl_khr_local_int32_extended_atomics cl_khr_byte_addressable_store cl_khr_depth_images cl_khr_3d_image_writes cl_intel_exec_by_local_thread cl_khr_spir cl_khr_fp64 

  Platform Name                                   NVIDIA CUDA
Number of devices                                 1
  Device Name                                     GeForce GT 720M
  Device Vendor                                   NVIDIA Corporation
  Device Vendor ID                                0x10de
  Device Version                                  OpenCL 1.1 CUDA
  Driver Version                                  340.96
  Device OpenCL C Version                         OpenCL C 1.1 
  Device Type                                     GPU
  Device Profile                                  FULL_PROFILE
  Device Topology (NV)                            PCI-E, 01:00.0
  Max compute units                               2
  Max clock frequency                             1250MHz
  Compute Capability (NV)                         2.1
  Max work item dimensions                        3
  Max work item sizes                             1024x1024x64
  Max work group size                             1024
  Preferred work group size multiple              32
  Warp size (NV)                                  32
  Preferred / native vector sizes                 
    char                                                 1 / 1       
    short                                                1 / 1       
    int                                                  1 / 1       
    long                                                 1 / 1       
    half                                                 0 / 0        (n/a)
    float                                                1 / 1       
    double                                               1 / 1        (cl_khr_fp64)
  Half-precision Floating-point support           (n/a)
  Single-precision Floating-point support         (core)
    Denormals                                     Yes
    Infinity and NANs                             Yes
    Round to nearest                              Yes
    Round to zero                                 Yes
    Round to infinity                             Yes
    IEEE754-2008 fused multiply-add               Yes
    Support is emulated in software               No
    Correctly-rounded divide and sqrt operations  No
  Double-precision Floating-point support         (cl_khr_fp64)
    Denormals                                     Yes
    Infinity and NANs                             Yes
    Round to nearest                              Yes
    Round to zero                                 Yes
    Round to infinity                             Yes
    IEEE754-2008 fused multiply-add               Yes
    Support is emulated in software               No
    Correctly-rounded divide and sqrt operations  No
  Address bits                                    32, Little-Endian
  Global memory size                              2147221504 (2GiB)
  Error Correction support                        No
  Max memory allocation                           536805376 (511.9MiB)
  Unified memory for Host and Device              No
  Integrated memory (NV)                          No
  Minimum alignment for any data type             128 bytes
  Alignment of base address                       4096 bits (512 bytes)
  Global Memory cache type                        Read/Write
  Global Memory cache size                        32768
  Global Memory cache line                        128 bytes
  Image support                                   Yes
    Max number of samplers per kernel             16
    Max 2D image size                             32768x32768 pixels
    Max 3D image size                             2048x2048x2048 pixels
    Max number of read image args                 128
    Max number of write image args                8
  Local memory type                               Local
  Local memory size                               49152 (48KiB)
  Registers per block (NV)                        32768
  Max constant buffer size                        65536 (64KiB)
  Max number of constant args                     9
  Max size of kernel argument                     4352 (4.25KiB)
  Queue properties                                
    Out-of-order execution                        Yes
    Profiling                                     Yes
  Profiling timer resolution                      1000ns
  Execution capabilities                          
    Run OpenCL kernels                            Yes
    Run native kernels                            No
    Kernel execution timeout (NV)                 Yes
  Concurrent copy and kernel execution (NV)       Yes
    Number of async copy engines                  1
  Device Available                                Yes
  Compiler Available                              Yes
  Device Extensions                               cl_khr_byte_addressable_store cl_khr_icd cl_khr_gl_sharing cl_nv_compiler_options cl_nv_device_attribute_query cl_nv_pragma_unroll cl_nv_copy_opts  cl_khr_global_int32_base_atomics cl_khr_global_int32_extended_atomics cl_khr_local_int32_base_atomics cl_khr_local_int32_extended_atomics cl_khr_fp64 

NULL platform behavior
  clGetPlatformInfo(NULL, CL_PLATFORM_NAME, ...)  No platform
  clGetDeviceIDs(NULL, CL_DEVICE_TYPE_ALL, ...)   No platform
  clCreateContext(NULL, ...) [default]            No platform
  clCreateContext(NULL, ...) [other]              Success [INTEL]
  clCreateContextFromType(NULL, CL_DEVICE_TYPE_CPU)  No platform
  clCreateContextFromType(NULL, CL_DEVICE_TYPE_GPU)  No platform
  clCreateContextFromType(NULL, CL_DEVICE_TYPE_ACCELERATOR)  No platform
  clCreateContextFromType(NULL, CL_DEVICE_TYPE_CUSTOM)  No platform
  clCreateContextFromType(NULL, CL_DEVICE_TYPE_ALL)  No platform
  \end{lstlisting}

  Here is a performance comparison between Intel CPU and NVIDIA GPU on my laptop. The input image is 1080p
  (1920 x 1080 = 2073600px $\approx$ 2Mpx) which is a common resolution nowadays for consumer displays and video
  recording devices.

  \begin{lstlisting}
$ time ./pfft -vp 0 colorful-pattern-backgrounds-1.png 
Using OpenCL platform: Intel(R) OpenCL
Using OpenCL device:        Intel(R) Core(TM) i5-3230M CPU @ 2.60GHz
Build succeeded.
Compilation started
Compilation done
Linking started
Linking done
Device build started
Device build done
Kernel <inputX> was not vectorized
Kernel <inputY> was not vectorized
Kernel <inputZ> was not vectorized
Kernel <outputX> was successfully vectorized (8)
Kernel <outputY> was successfully vectorized (8)
Kernel <outputZ> was successfully vectorized (8)
Done.
4 warnings generated.
4 warnings generated.

real  0m7.804s
user  0m8.964s
sys 0m0.088s
$ time ./pfft -vp 1 colorful-pattern-backgrounds-1.png 
Using OpenCL platform: NVIDIA CUDA
Using OpenCL device: GeForce GT 720M
Build succeeded.



real  0m2.305s
user  0m2.032s
sys 0m0.272s
  \end{lstlisting}

  One can see that GPU is approximately 3.8x faster than CPU on my laptop for this task.

  However, debugging an OpenCL program is quite a hassle. It seems CPU cannot handle a very large image
  ($\approx$10Mpx) and yields out of resources error (-5) while GPU handles this well. But GPU fails on a smaller image
  ($\approx$1.7Mpx) and yields segmentation fault while CPU handles it well. All of the test samples are included in
  test.tar.gz as well as the binary pfft compiled under Ubuntu 14.04 x86\_64.

  \begin{lstlisting}
$ time ./pfft -vp 0 Inside_the_Batad_rice_terraces.png 
Using OpenCL platform: Intel(R) OpenCL
Using OpenCL device:        Intel(R) Core(TM) i5-3230M CPU @ 2.60GHz
Build succeeded.
Compilation started
Compilation done
Linking started
Linking done
Device build started
Device build done
Kernel <inputX> was not vectorized
Kernel <inputY> was not vectorized
Kernel <inputZ> was not vectorized
Kernel <outputX> was successfully vectorized (8)
Kernel <outputY> was successfully vectorized (8)
Kernel <outputZ> was successfully vectorized (8)
Done.
4 warnings generated.
4 warnings generated.
clfftEnqueueTransform failed with code -5

real  0m5.797s
user  0m6.068s
sys 0m0.112s
$ time ./pfft -vp 1 19267159-Abstract-hand-drawn-tile-seamless-Colorful-pattern-Stock-Vector.png 
Using OpenCL platform: NVIDIA CUDA
Using OpenCL device: GeForce GT 720M
Build succeeded.


Segmentation fault

real  0m0.112s
user  0m0.080s
sys 0m0.028s
  \end{lstlisting}


  \section{Further improvements}

  By using CPU, GPU and possibly other computing devices simultaneously one achieve maximum speed possible. But some
  modification needs to be done to accomplish this including assigning different part of image to different computing
  devices. One can also use MPI or OpenMP to communicate between different devices to achieve even faster speed.


\end{document}
